\documentclass[14pt]{article}

% Basic packages
\usepackage{subfiles}
\usepackage{extsizes}
\usepackage[quiet]{fontspec}
\usepackage{unicode-math}
\usepackage[margin=2cm]{geometry}
\usepackage{hyperref}
\usepackage{fancyhdr, lastpage}
\usepackage{multirow, multicol}
\usepackage{longtable}
\usepackage{siunitx}
\usepackage{amsmath, amssymb}
\usepackage{soul}
\usepackage{float}
\usepackage{graphicx}
\usepackage{subcaption}
\usepackage{pdfpages}
\usepackage{changepage}

% Font config
\defaultfontfeatures{Scale=MatchUppercase}
\setmainfont{Libertinus Serif}
\setsansfont{Carlito}
\setmonofont{DejaVu Sans Mono}
\setmathfont{Libertinus Math}
\setul{1pt}{.4pt}

% Page layout config
\let\OLDTITLE\title
\renewcommand{\title}[1]{\OLDTITLE{#1}\newcommand{\thetitle}{#1}}
\pagestyle{fancy}
\lhead{\thetitle}
\cfoot{Page \thepage\ of \pageref{LastPage}}

% Hyperref config
\hypersetup{colorlinks=true, urlcolor=blue}
\urlstyle{same}

% Minted config
\usepackage[outputdir={\detokenize{OUTDIR}}]{minted}
\renewcommand{\theFancyVerbLine}{\sffamily
\textcolor[rgb]{0.4,0.4,0.4}{\small \arabic{FancyVerbLine}}}
\setminted{linenos,autogobble,obeytabs,frame=single,highlightcolor=yellow}
\renewcommand{\listingscaption}{Source}

% \usepackage[singlefile, pdf]{graphviz}

\title{Evolutionary Computation Lab III}
\author{Piotr Kaszubski 148283}
\date{Tuesday, November  7, 2023}

\begin{document}
\maketitle
\tableofcontents
\newpage

\section{Problem description}
We are given three columns of integers with a row for each node. The first two
columns contain \verb`x` and \verb`y` coordinates of the node positions in a
plane. The third column contains node costs.

\begin{enumerate}
	\item Select exactly 50\% of the nodes (if the number of nodes is odd we
		round the number of nodes to be selected up).
	\item Form a Hamiltonian cycle (closed path) through this set of nodes such
		that the sum of the total length of the path plus the total cost of the
		selected nodes is minimized. The distances between nodes are calculated
		as Euclidean distances rounded mathematically to integer values.
\end{enumerate}

The distance matrix should be calculated just after reading an instance and
then only the distance matrix (no nodes coordinates) should be accessed by
optimization methods to allow instances defined only by distance matrices.

\section{Corrections from previous reports}
While working on this report, I discovered a trivial but serious bug that
unfortunately affected the previous 2 reports: when exporting the found
solutions (for the visualization plots), I was \emph{maximizing} the score
instead of minimizing. That means all the plots you have seen thus far have
been the \emph{worst} solutions, not best. The scores in tables were unaffected
by this error. \\
Fixed in \href{https://github.com/RoyalDonkey/put-ec-tasks/commit/c2604bc7d0fa83ca02c61c3e38b789cc1ecae52a}{c2604bc}.

\vspace{4mm}
\noindent
I rerun all experiments from report 1 and 2 and pushed the new plots. If you
wish, you can examine them here:
\begin{itemize}
	\item \href{https://github.com/RoyalDonkey/put-ec-tasks/tree/c2604bc7d0fa83ca02c61c3e38b789cc1ecae52a/task1/results}{updated report 1 plots}
	\item \href{https://github.com/RoyalDonkey/put-ec-tasks/tree/c2604bc7d0fa83ca02c61c3e38b789cc1ecae52a/task2/results}{updated report 2 plots}
\end{itemize}

\newpage
\section{Pseudocode}

\subsection{InitMoves}
\begin{description}
	\item [Input] set of graph nodes \verb`G`, set of candidate nodes \verb`N`
	\item [Output] A set \verb`M` of all possible moves. Each move is comprised
		of the \emph{move type} (intra-route node swap, intra-route edge swap,
		or inter-route node swap) and \emph{indices} of the nodes/edges
		involved.
	\item [Steps]
		\begin{enumerate}\item []
			\item Initialize \verb`M` to an empty list.
			\item For \verb`i` = 1{\ldots}\verb`sizeof(G)`, do:
				\begin{enumerate}
					\item For \verb`j` = \verb`i`{\ldots}\verb`sizeof(G)`, do:
						\begin{enumerate}
							\item Add \verb`(i, j,` intra-route node swap\verb`)` move to \verb`M`.
							\item Add \verb`(i, j,` intra-route edge swap\verb`)` move to \verb`M`.
						\end{enumerate}
				\end{enumerate}
			\item For \verb`i` = 1{\ldots}\verb`sizeof(G)`, do:
				\begin{enumerate}
					\item For \verb`j` = \verb`1`{\ldots}\verb`sizeof(N)`, do:
						\begin{enumerate}
							\item Add \verb`(i, j,` inter-route node swap\verb`)` move to \verb`M`.
						\end{enumerate}
				\end{enumerate}
		\end{enumerate}
	\item [Observation]
		\begin{itemize}\item []
			\item Since moves are index-based, and all graphs we will be
				dealing with have an equal number of nodes, \verb`InitMoves()`
				always has the same return value. To simplify things, I will
				call it without parameters.
		\end{itemize}
\end{description}

\subsection{Greedy local search}
\begin{description}
	\item [Input] set of graph nodes \verb`G`, set of candidate nodes \verb`N`
	\item [Output] A set of graph nodes \verb`G'`$ ∈ ($\verb`G`$∪$\verb`N`), \emph{at least} as good as
		\verb`G` w.r.t.\ the objective function, where \verb`sizeof(G')` = \verb`sizeof(G)`.
	\item [Steps]
		\begin{enumerate}\item []
			\item Let \verb`M` = \verb`InitMoves()`.
			\item Let \verb`did_improve` = \verb`true`.
			\item While \verb`did_improve`, do:
				\begin{enumerate}
					\item Set \verb`did_improve` = \verb`false`.
					\item Let \verb`M'` be a copy of \verb`M`.
					\item While \verb`sizeof(M') > 0`, do:
						\begin{enumerate}
							\item Pop a random move from \verb`M'` into \verb`m`.
							\item Compute score delta \verb`d` of \verb`m`.
							\item If \verb`d < 0`, then:
								\begin{enumerate}
									\item Perform \verb`m`.
									\item Set \verb`did_improve` = \verb`true`.
									\item Break to outer loop.
								\end{enumerate}
						\end{enumerate}
				\end{enumerate}
		\end{enumerate}
\end{description}

\subsection{Steepest local search}
\begin{description}
	\item [Input] set of graph nodes \verb`G`, set of candidate nodes \verb`N`
	\item [Output] A set of graph nodes \verb`G'`$ ∈ ($\verb`G`$∪$\verb`N`),
		\emph{at least} as good as \verb`G` w.r.t.\ the objective function,
		where \verb`sizeof(G')` = \verb`sizeof(G)`.
	\item [Steps]
		\begin{enumerate}\item []
			\item Let \verb`did_improve` = \verb`true`.
			\item While \verb`did_improve`, do:
				\begin{enumerate}
					\item Set \verb`did_improve` = \verb`false`.
					\item Find a move \verb`m` from \verb`InitMoves()` with the lowest delta \verb`d`.
					\item If \verb`d < 0`, then:
						\begin{enumerate}
							\item Perform \verb`m`.
							\item Set \verb`did_improve` = \verb`true`.
						\end{enumerate}
				\end{enumerate}
		\end{enumerate}
	\item [Remarks]
		\begin{itemize}\item []
			\item I used \verb`InitMoves()` here for simplification, but the
				actual implementation does not waste resources on constructing
				a complete neighborhood and simply iterates through the
				possible moves one-at-a-time.
		\end{itemize}
\end{description}

\newpage
\section{Results}
\begin{longtable}[c]{|c|cc|}
	\hline
	\multirow{2}*{\textbf{ALG.}} & \textbf{TSPA} & \textbf{TSPC} \\
	 & \textbf{TSPB} & \textbf{TSPD} \\
	\hline
	\endfirsthead
	\hline
	\multirow{2}*{\textbf{ALG.}} & \textbf{TSPA} & \textbf{TSPC} \\
	 & \textbf{TSPB} & \textbf{TSPD} \\
	\hline
	\endhead

	\multirow{2}*{greedy-random} & 264,383 (240,873--292,180) & 214,934 (185,725--239,535) \\
	& 267,346 (240,224--297,124) & 218,750 (194,342--243,930) \\
	\hline
	\multirow{2}*{greedy-nn} & 87,649 (84,471--95,013) & 58,877 (56,304--63,304) \\
	& 79,304 (77,448--82,669) & 54,405 (50,335--59,846) \\
	\hline
	\multirow{2}*{greedy-cycle} & 77,069 (75,666--80,321) & 55,845 (53,226--58,876) \\
	& 70,685 (68,764--76,324) & 55,055 (50,409--60,077) \\
	\hline
	\hline

	\multirow{2}*{2-regret} & 103,385 (86,224--119,577) & 66,161 (58,778--71,821) \\
	& 97,807 (83,179--113,449) & 63,387 (53,396--69,441) \\
	\hline
	\multirow{2}*{wsc} & 87,553 (80,427--101,498) & 60,334 (54,306--66,962) \\
	& 82,983 (73,310--95,733) & 58,478 (51,420--66,263) \\
	\hline
	\hline

	\multirow{2}*{ls-greedy-random} & 77,819 (75,305--81,728) & 51,615 (48,989--54,563) \\
	& 71,093 (68,171--76,696) & 48,712 (46,240--53,155) \\
	\hline
	\multirow{2}*{ls-greedy-preset} & 74,819 (74,690--74,888) & 53,080 (53,080--53,080) \\
	& 67,955 (67,748--68,316) & 45,721 (44,698--46,494) \\
	\hline
	\multirow{2}*{ls-steepest-random} & 77,866 (75,315--81,017) & 51,371 (49,257--53,785) \\
	& 71,322 (68,623--76,002) & 48,234 (45,351--51,534) \\
	\hline
	\multirow{2}*{ls-steepest-preset} & 74,837 (74,837--74,837) & 53,080 (53,080--53,080) \\
	& 67,852 (67,852--67,852) & 45,556 (45,556--45,556) \\
	\hline
	\caption{Average, minimum and maximum scores of found solutions}
\end{longtable}

\begin{longtable}[c]{|c|cc|}
	\hline
	\multirow{2}*{\textbf{ALG.}} & \textbf{TSPA} & \textbf{TSPC} \\
	 & \textbf{TSPB} & \textbf{TSPD} \\
	\hline
	\endfirsthead
	\hline
	\multirow{2}*{\textbf{ALG.}} & \textbf{TSPA} & \textbf{TSPC} \\
	 & \textbf{TSPB} & \textbf{TSPD} \\
	\hline
	\endhead
	\multirow{2}*{ls-greedy-random} & 35.826 (31.185--43.658) & 36.884 (31.790--52.520) \\
	& 36.761 (31.362--43.052) & 38.939 (30.861--50.768) \\
	\hline
	\multirow{2}*{ls-greedy-preset} & 5.050 (3.089--11.703) & 2.471 (1.739--3.529) \\
	& 4.435 (3.041--8.031) & 8.352 (5.892--11.703) \\
	\hline
	\multirow{2}*{ls-steepest-random} & 82.528 (70.451--117.119) & 82.132 (72.866--96.202) \\
	& 84.666 (73.417--98.120) & 83.055 (68.014--94.368) \\
	\hline
	\multirow{2}*{ls-steepest-preset} & 7.415 (7.290--12.772) & 2.442 (2.429--2.567) \\
	& 6.738 (6.681--7.263) & 24.005 (23.762--30.697) \\
	\hline
	\caption{Average, minimum, maximum running times per instance (ms)}
\end{longtable}

\section{Visualizations}

\newcommand{\visualization}[3]{%
\begin{figure}[H]%
	\begin{adjustwidth}{0}{0}%
		\includegraphics[width=\linewidth]{results/best_#2_#1.pdf}%
	\end{adjustwidth}%
	\vspace{-12mm}%
	\caption{Best #2 solution to #1 (#3)}%
\end{figure}%
}

\subsection{TSPA.csv}
\visualization{TSPA}{ls-greedy-random}{75,305}
\visualization{TSPA}{ls-greedy-preset}{74,690}
\visualization{TSPA}{ls-steepest-random}{75,315}
\visualization{TSPA}{ls-steepest-preset}{74,837}

\subsection{TSPB.csv}
\visualization{TSPB}{ls-greedy-random}{68,171}
\visualization{TSPB}{ls-greedy-preset}{67,748}
\visualization{TSPB}{ls-steepest-random}{68,623}
\visualization{TSPB}{ls-steepest-preset}{67,852}

\subsection{TSPC.csv}
\visualization{TSPC}{ls-greedy-random}{48,989}
\visualization{TSPC}{ls-greedy-preset}{53,080}
\visualization{TSPC}{ls-steepest-random}{49,257}
\visualization{TSPC}{ls-steepest-preset}{53,080}

\subsection{TSPD.csv}
\visualization{TSPD}{ls-greedy-random}{46,240}
\visualization{TSPD}{ls-greedy-preset}{44,698}
\visualization{TSPD}{ls-steepest-random}{45,351}
\visualization{TSPD}{ls-steepest-preset}{45,556}

\newpage
\section{Source code}
The source code for all the experiments and this report is hosted on GitHub: \\
\url{https://github.com/RoyalDonkey/put-ec-tasks}

\section{Conclusions}
I'm generally pleased with the results -- local search methods managed to
improve solutions in almost all cases and in some by a considerable margin.

An interesting anomaly with \textbf{TSPC} happened; it appears that the preset
solution I took from nearest neighbor run of the first report is, by some
insane(?) happenstance, a local optimum. This has caused both greedy and
steepest local searches to not be able to find a single improvement,
understandably so. As a result, the random-initialized variants beat the
greedy-initialized one, which in turn have equal average-min-max values.%
\footnote{It was later brought to my attention that I probably misunderstood
	the assignment a bit, and we were supposed to start local search 200 times
	from graphs that were obtained from a greedy method being started from
	different nodes. I instead read it as starting the \emph{local search} each
	time from a different node, which doesn't make a lot of sense and I chose
	to ignore this requirement entirely. I always start all preset local
	searches from the same solutions. My apologies, if I wasn't already past
	the schedule I would correct this. On the flip side, this gave me the
opportunity to notice and discuss the local optimum situation, which is pretty
interesting.}

\vspace{4mm}
\noindent
Similarly to report 2, I ended up implementing a data structure for this
assignment, which I ended up not using at all. I really need to stop doing
that.

I thought a min-heap would be handy for steepest neighborhood scan, because I
figured that it's a huge waste to rescan everything every iteration, when we're
only changing tiny parts of the graph at a time. This idea is good, but then I
forgot about it, ended up doing it the naïve way, finished the
report\footnote{FYI, the running time for all the experiments in this report
was about 1:53 minutes.}, and once I was past the finish line, I did not feel
like coming back to optimize it anymore.

\end{document}
