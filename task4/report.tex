\documentclass[14pt]{article}

% Basic packages
\usepackage{subfiles}
\usepackage{extsizes}
\usepackage[quiet]{fontspec}
\usepackage{unicode-math}
\usepackage[margin=2cm]{geometry}
\usepackage{hyperref}
\usepackage{fancyhdr, lastpage}
\usepackage{multirow, multicol}
\usepackage{longtable}
\usepackage{siunitx}
\usepackage{amsmath, amssymb}
\usepackage{soul}
\usepackage{float}
\usepackage{graphicx}
\usepackage{subcaption}
\usepackage{pdfpages}
\usepackage{changepage}

% Font config
\defaultfontfeatures{Scale=MatchUppercase}
\setmainfont{Libertinus Serif}
\setsansfont{Carlito}
\setmonofont{DejaVu Sans Mono}
\setmathfont{Libertinus Math}
\setul{1pt}{.4pt}

% Page layout config
\let\OLDTITLE\title
\renewcommand{\title}[1]{\OLDTITLE{#1}\newcommand{\thetitle}{#1}}
\pagestyle{fancy}
\lhead{\thetitle}
\cfoot{Page \thepage\ of \pageref{LastPage}}

% Hyperref config
\hypersetup{colorlinks=true, urlcolor=blue}
\urlstyle{same}

% Minted config
\usepackage[outputdir={\detokenize{OUTDIR}}]{minted}
\renewcommand{\theFancyVerbLine}{\sffamily
\textcolor[rgb]{0.4,0.4,0.4}{\small \arabic{FancyVerbLine}}}
\setminted{linenos,autogobble,obeytabs,frame=single,highlightcolor=yellow}
\renewcommand{\listingscaption}{Source}

% \usepackage[singlefile, pdf]{graphviz}

\title{Evolutionary Computation Lab IV}
\author{Piotr Kaszubski 148283}
\date{Monday, November 13, 2023}

\begin{document}
\maketitle
\tableofcontents
\newpage

\section{Problem description}
We are given three columns of integers with a row for each node. The first two
columns contain \verb`x` and \verb`y` coordinates of the node positions in a
plane. The third column contains node costs.

\begin{enumerate}
	\item Select exactly 50\% of the nodes (if the number of nodes is odd we
		round the number of nodes to be selected up).
	\item Form a Hamiltonian cycle (closed path) through this set of nodes such
		that the sum of the total length of the path plus the total cost of the
		selected nodes is minimized. The distances between nodes are calculated
		as Euclidean distances rounded mathematically to integer values.
\end{enumerate}

The distance matrix should be calculated just after reading an instance and
then only the distance matrix (no nodes coordinates) should be accessed by
optimization methods to allow instances defined only by distance matrices.

\section{Pseudocode}
The method is mostly the same as in the previous report. However, there is an
extra preprocessing step and a slight alteration to the runtime logic.

\subsection{Before linear search}
\begin{enumerate}
	\item Initialize an \verb`NxN` boolean matrix \verb`M` to \verb`false`
		values, where \verb`N` is the number of nodes in the graph \verb`G`.
	\item For each node \verb`n` in \verb`G`, do:
		\begin{enumerate}
			\item Find 10 nodes with the smallest distance to \verb`n`
				(excluding \verb`n`), and mark them as \verb`true` in \verb`M`.
		\end{enumerate}
\end{enumerate}

Once we have the boolean matrix of 10 closest nodes to each node, we use it to
cache all acceptable moves (of which there are 3 types: inter-route node swap,
intra-route node swap, intra-route edge swap).

\noindent
Inside the linear search algorithm, we modify:
\begin{enumerate}
	\item For index \verb`i` in \verb`size(G)`, do:
		\begin{enumerate}
			\item For index \verb`j` in \verb`size(G)`, do:
				\begin{enumerate}
					\item Evaluate intra-route node swap between \verb`i`th and
						\verb`j`th node.
					\item Evaluate intra-route edge swap between \verb`i`th and
						\verb`j`th node.
				\end{enumerate}
		\end{enumerate}
	\item For index \verb`i` in \verb`size(G)`, do:
		\begin{enumerate}
			\item For index \verb`j` in \verb`size(G')` (where \verb`G'` is the set of nodes outside of \verb`G`), do:
				\begin{enumerate}
					\item Evaluate inter-route node swap between \verb`i`th
						node in \verb`G` and \verb`j`th node in \verb`G'`.
				\end{enumerate}
		\end{enumerate}
\end{enumerate}
\ldots to instead be:
\begin{enumerate}
	\item For index \verb`i` in \verb`size(G)`, do:
		\begin{enumerate}
			\item For index \verb`j` in \verb`size(G)`, do:
				\begin{enumerate}
					\item \textbf{If intra-route node swap for} \verb`(i, j)` \textbf{is cached as acceptable, do}:
						\begin{enumerate}
							\item Evaluate intra-route node swap between \verb`i`th and
								\verb`j`th node.
						\end{enumerate}
					\item \textbf{If intra-route edge swap for} \verb`(i, j)` \textbf{is cached as acceptable, do}:
						\begin{enumerate}
							\item Evaluate intra-route edge swap between \verb`i`th and
								\verb`j`th node.
						\end{enumerate}
				\end{enumerate}
		\end{enumerate}
	\item For index \verb`i` in \verb`size(G)`, do:
		\begin{enumerate}
			\item For index \verb`j` in \verb`size(G')` (where \verb`G'` is the set of nodes outside of \verb`G`), do:
				\begin{enumerate}
					\item \textbf{If inter-route node swap for} \verb`(i, j)` \textbf{is cached as acceptable, do}:
						\begin{enumerate}
							\item Evaluate inter-route node swap between \verb`i`th
								node in \verb`G` and \verb`j`th node in \verb`G'`.
						\end{enumerate}
				\end{enumerate}
		\end{enumerate}
\end{enumerate}

\newpage
\section{Results}
\begin{longtable}[c]{|c|cc|}
	\hline
	\multirow{2}*{\textbf{ALG.}} & \textbf{TSPA} & \textbf{TSPC} \\
	 & \textbf{TSPB} & \textbf{TSPD} \\
	\hline
	\endfirsthead
	\hline
	\multirow{2}*{\textbf{ALG.}} & \textbf{TSPA} & \textbf{TSPC} \\
	 & \textbf{TSPB} & \textbf{TSPD} \\
	\hline
	\endhead

	\multirow{2}*{ls-steepest-random} & 77,866 (75,315--81,017) & 51,371 (49,257--53,785) \\
	& 71,322 (68,623--76,002) & 48,234 (45,351--51,534) \\
	\hline

	\multirow{2}*{lsc-steepest-random} & 89,197 (83,559--99,633) & 62,674 (56,990--74,215) \\
	 & 84,469 (74,393--96,069) & 60,289 (53,205--67,824) \\
	\hline

	\caption{Average, minimum and maximum scores of found solutions}
\end{longtable}

\begin{longtable}[c]{|c|cc|}
	\hline
	\multirow{2}*{\textbf{ALG.}} & \textbf{TSPA} & \textbf{TSPC} \\
	 & \textbf{TSPB} & \textbf{TSPD} \\
	\hline
	\endfirsthead
	\hline
	\multirow{2}*{\textbf{ALG.}} & \textbf{TSPA} & \textbf{TSPC} \\
	 & \textbf{TSPB} & \textbf{TSPD} \\
	\hline
	\endhead
	\multirow{2}*{ls-steepest-random} & 82.528 (70.451--117.119) & 82.132 (72.866--96.202) \\
	& 84.666 (73.417--98.120) & 83.055 (68.014--94.368) \\
	\hline
	\multirow{2}*{lsc-steepest-preset} & 19.134 (14.749--25.742) & 18.321 (13.988--24.015) \\
	& 18.863 (14.981--25.721) & 17.899 (14.340--22.888) \\
	\hline
	\caption{Average, minimum, maximum running times per instance (ms)}
\end{longtable}

\newpage
\section{Visualizations}

\newcommand{\visualization}[3]{%
\begin{figure}[H]%
	\begin{adjustwidth}{0}{0}%
		\includegraphics[width=\linewidth]{results/best_#2_#1.pdf}%
	\end{adjustwidth}%
	\vspace{-12mm}%
	\caption{Best #2 solution to #1 (#3)}%
\end{figure}%
}

\subsection{TSPA.csv}
\visualization{TSPA}{ls-steepest-random}{75,315}
\visualization{TSPA}{lsc-steepest-random}{83,559}

\subsection{TSPB.csv}
\visualization{TSPB}{ls-steepest-random}{68,623}
\visualization{TSPB}{lsc-steepest-random}{74,393}

\subsection{TSPC.csv}
\visualization{TSPC}{ls-steepest-random}{49,257}
\visualization{TSPC}{lsc-steepest-random}{56,990}

\subsection{TSPD.csv}
\visualization{TSPD}{ls-steepest-random}{45,351}
\visualization{TSPD}{lsc-steepest-random}{53,205}

\section{Source code}
The source code for all the experiments and this report is hosted on GitHub: \\
\url{https://github.com/RoyalDonkey/put-ec-tasks}

\section{Conclusions}
Candidate moves are a nice way to introduce a parameterized trade-off between
the accuracy of the search and the running time. I tested the experiments with
\verb`n=15` as well, and it extended the time from around 15 seconds to 22
seconds, but managed to find solutions better by about 3,000.

It is visible from the plots that the candidate method settled for more
high-cost nodes. I wonder if the tendency to pick longer edges is
proportionally the same (it is difficult to tell from the plots, because edge
lengths are not as easy to compare).
\end{document}
