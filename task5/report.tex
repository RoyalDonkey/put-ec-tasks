\documentclass[14pt]{article}

% Basic packages
\usepackage{subfiles}
\usepackage{extsizes}
\usepackage[quiet]{fontspec}
\usepackage{unicode-math}
\usepackage[margin=2cm]{geometry}
\usepackage{hyperref}
\usepackage{fancyhdr, lastpage}
\usepackage{multirow, multicol}
\usepackage{longtable}
\usepackage{siunitx}
\usepackage{amsmath, amssymb}
\usepackage{soul}
\usepackage{float}
\usepackage{graphicx}
\usepackage{subcaption}
\usepackage{pdfpages}
\usepackage{changepage}

% Font config
\defaultfontfeatures{Scale=MatchUppercase}
\setmainfont{Libertinus Serif}
\setsansfont{Carlito}
\setmonofont{DejaVu Sans Mono}
\setmathfont{Libertinus Math}
\setul{1pt}{.4pt}

% Page layout config
\let\OLDTITLE\title
\renewcommand{\title}[1]{\OLDTITLE{#1}\newcommand{\thetitle}{#1}}
\pagestyle{fancy}
\lhead{\thetitle}
\cfoot{Page \thepage\ of \pageref{LastPage}}

% Hyperref config
\hypersetup{colorlinks=true, urlcolor=blue}
\urlstyle{same}

% Minted config
\usepackage[outputdir={\detokenize{OUTDIR}}]{minted}
\renewcommand{\theFancyVerbLine}{\sffamily
\textcolor[rgb]{0.4,0.4,0.4}{\small \arabic{FancyVerbLine}}}
\setminted{linenos,autogobble,obeytabs,frame=single,highlightcolor=yellow}
\renewcommand{\listingscaption}{Source}

% \usepackage[singlefile, pdf]{graphviz}

\title{Evolutionary Computation Lab V}
\author{Piotr Kaszubski 148283}
\date{}

\begin{document}
\maketitle
\tableofcontents
\newpage

\section{Problem description}
We are given three columns of integers with a row for each node. The first two
columns contain \verb`x` and \verb`y` coordinates of the node positions in a
plane. The third column contains node costs.

\begin{enumerate}
	\item Select exactly 50\% of the nodes (if the number of nodes is odd we
		round the number of nodes to be selected up).
	\item Form a Hamiltonian cycle (closed path) through this set of nodes such
		that the sum of the total length of the path plus the total cost of the
		selected nodes is minimized. The distances between nodes are calculated
		as Euclidean distances rounded mathematically to integer values.
\end{enumerate}

The distance matrix should be calculated just after reading an instance and
then only the distance matrix (no nodes coordinates) should be accessed by
optimization methods to allow instances defined only by distance matrices.

\section{Pseudocode}

\section{Visualizations}

\newcommand{\visualization}[3]{%
\begin{figure}[H]%
	\begin{adjustwidth}{0}{0}%
		\includegraphics[width=\linewidth]{results/best_#2_#1.pdf}%
	\end{adjustwidth}%
	\vspace{-12mm}%
	\caption{Best #2 solution to #1 (#3)}%
\end{figure}%
}

% \subsection{TSPA.csv}
% \visualization{TSPA}{ls-steepest-random}{75,315}
% \visualization{TSPA}{lsc-steepest-random}{83,559}

\section{Source code}
The source code for all the experiments and this report is hosted on GitHub: \\
\url{https://github.com/RoyalDonkey/put-ec-tasks}

\section{Conclusions}
