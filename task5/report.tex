\documentclass[14pt]{article}

% Basic packages
\usepackage{subfiles}
\usepackage{extsizes}
\usepackage[quiet]{fontspec}
\usepackage{unicode-math}
\usepackage[margin=2cm]{geometry}
\usepackage{hyperref}
\usepackage{fancyhdr, lastpage}
\usepackage{multirow, multicol}
\usepackage{longtable}
\usepackage{siunitx}
\usepackage{amsmath, amssymb}
\usepackage{soul}
\usepackage{float}
\usepackage{graphicx}
\usepackage{subcaption}
\usepackage{pdfpages}
\usepackage{changepage}

% Font config
\defaultfontfeatures{Scale=MatchUppercase}
\setmainfont{Libertinus Serif}
\setsansfont{Carlito}
\setmonofont{DejaVu Sans Mono}
\setmathfont{Libertinus Math}
\setul{1pt}{.4pt}

% Page layout config
\let\OLDTITLE\title
\renewcommand{\title}[1]{\OLDTITLE{#1}\newcommand{\thetitle}{#1}}
\pagestyle{fancy}
\lhead{\thetitle}
\cfoot{Page \thepage\ of \pageref{LastPage}}

% Hyperref config
\hypersetup{colorlinks=true, urlcolor=blue}
\urlstyle{same}

% Minted config
\usepackage[outputdir={\detokenize{OUTDIR}}]{minted}
\renewcommand{\theFancyVerbLine}{\sffamily
\textcolor[rgb]{0.4,0.4,0.4}{\small \arabic{FancyVerbLine}}}
\setminted{autogobble,obeytabs,frame=single,highlightcolor=yellow}
\renewcommand{\listingscaption}{Source}

% \usepackage[singlefile, pdf]{graphviz}

\title{Evolutionary Computation Lab V}
\author{Piotr Kaszubski 148283}
\date{Sunday, November 26, 2023}

\begin{document}
\maketitle
\tableofcontents
\newpage

\section{Problem description}
We are given three columns of integers with a row for each node. The first two
columns contain \verb`x` and \verb`y` coordinates of the node positions in a
plane. The third column contains node costs.

\begin{enumerate}
	\item Select exactly 50\% of the nodes (if the number of nodes is odd we
		round the number of nodes to be selected up).
	\item Form a Hamiltonian cycle (closed path) through this set of nodes such
		that the sum of the total length of the path plus the total cost of the
		selected nodes is minimized. The distances between nodes are calculated
		as Euclidean distances rounded mathematically to integer values.
\end{enumerate}

The distance matrix should be calculated just after reading an instance and
then only the distance matrix (no nodes coordinates) should be accessed by
optimization methods to allow instances defined only by distance matrices.

\section{Corrections from previous reports}

\subsubsection*{Incorrect inter-swapping in reports 3 and 4}
I discovered that I was incorrectly passing parameters to a function
responsible for inter-swapping nodes. As a result, all of the inter-swaps were
semi-random. It was an easy fix, and it didn't change the results much (mostly
there were improvements by up to a thousand, in some cases the result got worse
by up to a thousand). The results in this report have been updated. \\
Fixed in \href{https://github.com/RoyalDonkey/put-ec-tasks/commit/1da9927b88eeb4440252d694a406773ccee52da2}{1da9927}.

\section{Pseudocode}
The logic behind the delta cache is rather complicated, so I think it's better,
to our own benefit, if I explain the \emph{strategy}, rather than transcribing
tons of functions to a not much more legible form here.

\begin{itemize}
	\item The cache is comprised of 3 \verb`N`$\times$\verb`N` matrices (\verb`N`
		being the number of nodes in the solution), one for:
		\begin{itemize}
			\item \textbf{inter-route node swaps}
			\item \textbf{intra-route node swaps}
			\item \textbf{intra-route edge swaps}
		\end{itemize}
		These matrices are indexed by node IDs, and for each pair of nodes
		store a cached delta, or an empty value (internally represented as
		\verb`LONG_MIN`).
	\item Whenever we need to compute a delta for any type of swap, we check if
		a cached value exists in the corresponding matrix. If so, we return it
		without doing any of the regular computation. Otherwise, after the
		computation, we store the new delta in the matrix.
	\item[]
		\begin{itemize}
			\item After an inter-route node swap, \emph{update}:
				\begin{itemize}
					\item the node that just got added to the graph,
					\item its two neighboring nodes.
				\end{itemize}
			\item After an intra-route node swap, \emph{update}:
				\begin{itemize}
					\item the two nodes that got swapped,
					\item their neighboring nodes (four nodes in total).
				\end{itemize}
			\item After an intra-route edge swap, \emph{update}:
				\begin{itemize}
					\item all nodes within the segment that got "reversed",
					\item the neighboring nodes of the segment (two in total).%
						\footnote{I tried making it work with only 4 node
						updates for the segment endpoints, but it didn't work out.}
				\end{itemize}
		\end{itemize}
	\item An \emph{update} operation for a node is defined as follows:
		\begin{itemize}
			\item For the inter-route node swap cache matrix:
				\begin{itemize}
					\item Recompute non-empty deltas between the target node
						and all nodes currently \textbf{outside} of the graph.
				\end{itemize}
				\begin{itemize}
					\item Delete deltas between the target node and all nodes
						currently \textbf{inside} the graph.%
						\footnote{
							It wasn't immediately obvious to me why this is
							needed, but not doing it caused inconsistencies to
							emerge in the matrix over time. It makes sense --
							an inter-route swap between two nodes inside the
							graph is an illegal operation.
						}
				\end{itemize}
			\item For the intra-route node and edge swap cache matrices:
				\begin{itemize}
					\item Recompute non-empty deltas between the target node
						and all nodes currently \textbf{inside} the graph.
				\end{itemize}
		\end{itemize}
\end{itemize}

\newpage
\section{Results}
\begin{longtable}[c]{|c|cc|}
	\hline
	\multirow{2}*{\textbf{ALG.}} & \textbf{TSPA} & \textbf{TSPC} \\
	& \textbf{TSPB} & \textbf{TSPD} \\
	\hline
	\endfirsthead
	\hline
	\multirow{2}*{\textbf{ALG.}} & \textbf{TSPA} & \textbf{TSPC} \\
	& \textbf{TSPB} & \textbf{TSPD} \\
	\hline
	\endhead
	\multirow{2}*{ls-steepest-random} & 77,866 (75,315--81,017) & 51,453 (49,257--53,785) \\
	& 71,322 (68,623--76,002) & 48,234 (45,351--51,534) \\
	\hline
	\multirow{2}*{lsc-steepest-random} & 89,127 (82,350--101,152) & 62,892 (55,074--74,956) \\
	& 84,469 (74,393--96,069) & 60,203 (51,897--67,824) \\
	\hline
	\multirow{2}*{lsd-steepest-random} & 77,866 (75,315--81,017) & 51,453 (49,257--53,785) \\
	& 71,322 (68,623--76,002) & 48,234 (45,351--51,534) \\
	\hline
	\multirow{2}*{lscd-steepest-random} & 89,127 (82,350--101,152) & 62,892 (55,074--74,956) \\
	& 84,469 (74,393--96,069) & 60,203 (51,897--67,824) \\
	\hline
	\caption{Average, minimum and maximum scores of found solutions}
\end{longtable}

\begin{longtable}[c]{|c|cc|}
	\hline
	\multirow{2}*{\textbf{ALG.}} & \textbf{TSPA} & \textbf{TSPC} \\
	& \textbf{TSPB} & \textbf{TSPD} \\
	\hline
	\endfirsthead
	\hline
	\multirow{2}*{\textbf{ALG.}} & \textbf{TSPA} & \textbf{TSPC} \\
	& \textbf{TSPB} & \textbf{TSPD} \\
	\hline
	\endhead
	\multirow{2}*{ls-steepest-random} & 85.737 (71.796--105.250) & 85.470 (72.226--98.981) \\
	& 88.181 (73.757--107.447) & 87.450 (67.483--102.099) \\
	\hline
	\multirow{2}*{lsc-steepest-random} & 21.973 (17.070--28.766) & 20.936 (15.504--30.904) \\
	& 22.002 (17.069--29.329) & 21.179 (15.651--28.530) \\
	\hline
	\multirow{2}*{lsd-steepest-random} & 43.684 (33.639--66.647) & 42.885 (33.038--57.912) \\
	& 43.794 (35.071--54.709) & 43.747 (30.598--74.864) \\
	\hline
	\multirow{2}*{lscd-steepest-random} & 28.267 (18.670--39.157) & 27.632 (18.752--39.882) \\
	& 28.695 (19.072--38.583) & 27.669 (19.578--40.003) \\
	\hline
	\caption{Average, minimum, maximum running times per instance (ms)}
\end{longtable}

\section{Visualizations}

\newcommand{\visualization}[3]{%
\begin{figure}[H]%
	\begin{adjustwidth}{0}{0}%
		\includegraphics[width=\linewidth]{results/best_#2_#1.pdf}%
	\end{adjustwidth}%
	\vspace{-12mm}%
	\caption{Best #2 solution to #1 (#3)}%
\end{figure}%
}

\subsection{TSPA.csv}
\visualization{TSPA}{lsd-steepest-random}{75,315}
\visualization{TSPA}{lscd-steepest-random}{82,350}

\subsection{TSPB.csv}
\visualization{TSPB}{lsd-steepest-random}{68,623}
\visualization{TSPB}{lscd-steepest-random}{74,393}

\subsection{TSPC.csv}
\visualization{TSPC}{lsd-steepest-random}{49,257}
\visualization{TSPC}{lscd-steepest-random}{55,074}

\subsection{TSPD.csv}
\visualization{TSPD}{lsd-steepest-random}{45,351}
\visualization{TSPD}{lscd-steepest-random}{51,897}

\section{Source code}
The source code for all the experiments and this report is hosted on GitHub: \\
\url{https://github.com/RoyalDonkey/put-ec-tasks}

\section{Conclusions}
To address the elephant in the room, my implementation of delta cache speeds
the algorithm \emph{without} candidate moves, but slows down the one
\emph{with} candidate moves. This is a surprising result, so I profiled my
programs with \verb`gprof` and I'm including the first few lines from their
flat profiles below:

\subsubsection*{\texttt{ls-steepest-random}}
\vspace{-6mm}
\begin{listing}[H]
	\begin{minted}[fontsize=\footnotesize, highlightlines={3-5}]{text}
		  %   cumulative   self              self     total
		time   seconds   seconds    calls   s/call   s/call  name
		34.43    20.36    20.36 1079710000   0.00     0.00  tsp_graph_evaluate_inter_swap
		33.39    40.11    19.75  545253550   0.00     0.00  tsp_nodes_evaluate_swap_nodes
		16.38    49.79     9.68  545253550   0.00     0.00  tsp_nodes_evaluate_swap_edges
		8.20     54.64     4.85                             sp_stack_get
		3.59     56.77     2.12        800   0.00     0.07  lsearch_steepest
		3.48     58.83     2.06                             sp_stack_peek
		0.58     59.17     0.34      33694   0.00     0.00  tsp_nodes_swap_edges
		0.00     59.17     0.00     512823   0.00     0.00  tsp_nodes_swap_nodes
	\end{minted}
\end{listing}
\vspace{-15mm}
\subsubsection*{\texttt{lsd-steepest-random}}
\vspace{-6mm}
\begin{listing}[H]
	\begin{minted}[fontsize=\footnotesize, highlightlines={4-5,10}]{text}
		  %   cumulative   self              self     total
		time   seconds   seconds    calls   s/call   s/call  name
		16.03     5.19     5.19 1079710000   0.00     0.00  tsp_graph_evaluate_inter_swap_with_delta_cache
		15.90    10.34     5.15  140098289   0.00     0.00  tsp_nodes_evaluate_swap_nodes
		10.28    13.68     3.33  148181901   0.00     0.00  tsp_graph_evaluate_inter_swap
		10.04    16.93     3.25  545253550   0.00     0.00  tsp_graph_evaluate_swap_nodes_with_delta_cache
		9.88     20.13     3.20                             sp_stack_get
		8.83     22.99     2.86        800   0.00     0.03  lsearch_delta_steepest
		8.60     25.78     2.79  545253550   0.00     0.00  tsp_graph_evaluate_swap_edges_with_delta_cache
		7.32     28.15     2.37  140098289   0.00     0.00  tsp_nodes_evaluate_swap_edges
		5.96     30.08     1.93    1361338   0.00     0.00  tsp_graph_update_delta_cache_for_node
		5.81     31.96     1.88                             sp_stack_peek
	\end{minted}
\end{listing}
\noindent
For local search without candidate moves, the time spent inside evaluation
functions dropped from 50 seconds to only about 11 seconds!

\subsubsection*{\texttt{lsc-steepest-random}}
\vspace{-6mm}
\begin{listing}[H]
	\begin{minted}[fontsize=\footnotesize, highlightlines={3-4,6}]{text}
		  %   cumulative   self              self     total
		time   seconds   seconds    calls   s/call   s/call  name
		33.42     4.51     4.51 105882052    0.00     0.00  tsp_nodes_evaluate_swap_nodes
		21.27     7.38     2.87 112972806    0.00     0.00  tsp_graph_evaluate_inter_swap
		19.79    10.05     2.67       800    0.00     0.02  lsearch_candidates_steepest
		5.78     10.83     0.78  56845336    0.00     0.00  tsp_nodes_evaluate_swap_edges
		5.26     11.55     0.71                             sp_stack_get
		4.08     12.10     0.55  31840000    0.00     0.00  tsp_heap_push
		1.89     12.35     0.26   8000000    0.00     0.00  tsp_nodes_swap_nodes_adds_candidate
		1.70     12.58     0.23         1    0.23    12.56  run_lsearch_algorithm
	\end{minted}
\end{listing}
\vspace{-15mm}
\subsubsection*{\texttt{lscd-steepest-random}}
\vspace{-6mm}
\begin{listing}[H]
	\begin{minted}[fontsize=\footnotesize, highlightlines={3,6-7}]{text}
		  %   cumulative   self              self     total
		time   seconds   seconds    calls   s/call   s/call  name
		22.64     3.66     3.66  98267679    0.00     0.00  tsp_nodes_evaluate_swap_nodes
		14.23     5.96     2.30   1311342    0.00     0.00  tsp_graph_update_delta_cache_for_node
		10.95     7.73     1.77       800    0.00     0.02  lsearch_candidates_delta_steepest
		9.59      9.28     1.55  63986193    0.00     0.00  tsp_graph_evaluate_inter_swap
		7.92     10.56     1.28  74051064    0.00     0.00  tsp_nodes_evaluate_swap_edges
		6.77     11.66     1.10  112972806   0.00     0.00  tsp_graph_evaluate_inter_swap_with_delta_cache
		6.68     12.74     1.08                             sp_stack_get
		5.07     13.56     0.82  105882052   0.00     0.00  tsp_graph_evaluate_swap_nodes_with_delta_cache
	\end{minted}
\end{listing}
\noindent
With candidate moves enabled, far less time is spent evaluating (about 8
seconds). Caching brings that down to 6.5 seconds, but the gain is simply
too small to outweigh the cost of caching.

\paragraph{}
This report proved exceptionally difficult to finish, so it's good we got 2
weeks for it. I'm not completely happy with my implementation, because I had to
cut corners in a few places due to lack of time. It's possible that my code
does more updating than strictly needed, thus slowing it down.

Still, I am glad about the results, especially after doing the profiler
breakdown.

\end{document}
