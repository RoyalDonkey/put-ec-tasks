\documentclass[14pt]{article}

% Basic packages
\usepackage{subfiles}
\usepackage{extsizes}
\usepackage[quiet]{fontspec}
\usepackage{unicode-math}
\usepackage[margin=2cm]{geometry}
\usepackage{hyperref}
\usepackage[nameinlink]{cleveref}
\usepackage{fancyhdr, lastpage}
\usepackage{multirow, multicol}
\usepackage{longtable}
\usepackage{siunitx}
\usepackage{amsmath, amssymb}
\usepackage{soul}
\usepackage{float}
\usepackage{graphicx}
\usepackage{subcaption}
\usepackage{pdfpages}
\usepackage{changepage}

% Font config
\defaultfontfeatures{Scale=MatchUppercase}
\setmainfont{Libertinus Serif}
\setsansfont{Carlito}
\setmonofont{DejaVu Sans Mono}
\setmathfont{Libertinus Math}
\setul{1pt}{.4pt}

% Page layout config
\let\OLDTITLE\title
\renewcommand{\title}[1]{\OLDTITLE{#1}\newcommand{\thetitle}{#1}}
\pagestyle{fancy}
\lhead{\thetitle}
\cfoot{Page \thepage\ of \pageref{LastPage}}

% Hyperref config
\hypersetup{colorlinks=true, urlcolor=blue}
\urlstyle{same}

% Minted config
\usepackage[outputdir={\detokenize{OUTDIR}}]{minted}
\renewcommand{\theFancyVerbLine}{\sffamily
\textcolor[rgb]{0.4,0.4,0.4}{\small \arabic{FancyVerbLine}}}
\setminted{autogobble,obeytabs,frame=single,highlightcolor=yellow}
\renewcommand{\listingscaption}{Source}

% \usepackage[singlefile, pdf]{graphviz}

\title{Evolutionary Computation Lab IX}
\author{Piotr Kaszubski 148283}
\date{Monday, January  8, 2024}

\begin{document}
\maketitle
\tableofcontents
\newpage

\section{Problem description}
We are given three columns of integers with a row for each node. The first two
columns contain \verb`x` and \verb`y` coordinates of the node positions in a
plane. The third column contains node costs.

\begin{enumerate}
	\item Select exactly 50\% of the nodes (if the number of nodes is odd we
		round the number of nodes to be selected up).
	\item Form a Hamiltonian cycle (closed path) through this set of nodes such
		that the sum of the total length of the path plus the total cost of the
		selected nodes is minimized. The distances between nodes are calculated
		as Euclidean distances rounded mathematically to integer values.
\end{enumerate}

The distance matrix should be calculated just after reading an instance and
then only the distance matrix (no nodes coordinates) should be accessed by
optimization methods to allow instances defined only by distance matrices.

\section{Pseudocode}
\subsection{Recombination operators}
Both recombination operators start by initializing an empty child and adding
\textbf{Common Edges} from both parents to it. Then, the two options are:
\begin{description}
	\item [op1] -- fill the remaining nodes randomly,
	\item [op2] -- fill the remaining nodes using the \emph{greedy cycle}
		strategy from report 1 (also used in report 7).
\end{description}

\paragraph{Common Edges algorithm} The algorithm is based on Edge Recombination
Crossover (ERX) from the lectures, with a small modification:
\begin{enumerate}
	\item Select a random node from either parent (that is not already present
		in the child). If there are no such nodes, terminate.
	\item Select subsequent elements randomly from a set of all nodes which
		\textbf{share an edge} with the last selected element in either of the
		parents. If there are no such nodes, go back to 1.
\end{enumerate}

\paragraph{Difference with ERX} The lecture slides mention only the
"subsequent" element, which suggests we consider only one of two adjacent nodes
per parent. I chose to relax this mechanism, allowing up to 4 continuations per
iteration, instead of 2. It seemed like a good idea, because a cycle of nodes
is bidirectional. But I don't know if it actually makes things better or worse.
At the very least it should reduce the likelihood of premature convergence, I
suppose.

\subsection{Runtime constraint}

To keep comparisons consistent with previous algorithms, I set the same CPU
time limit to all the evolutionary methods presented in this report (16.323 CPU
seconds).

\section{Results}

\begin{longtable}[c]{|c|cc|}
	\hline
	\multirow{2}*{\textbf{ALG.}} & \textbf{TSPA} & \textbf{TSPC} \\
	& \textbf{TSPB} & \textbf{TSPD} \\
	\hline
	\endfirsthead
	\hline
	\multirow{2}*{\textbf{ALG.}} & \textbf{TSPA} & \textbf{TSPC} \\
	& \textbf{TSPB} & \textbf{TSPD} \\
	\hline
	\endhead
	\multirow{2}*{ls-steepest-random} & 77,866 (75,315--81,017) & 51,453 (49,257--53,785) \\
	& 71,322 (68,623--76,002) & 48,234 (45,351--51,534) \\
	\hline
	\multirow{2}*{lsc-steepest-random} & 89,127 (82,350--101,152) & 62,892 (55,074--74,956) \\
	& 84,469 (74,393--96,069) & 60,203 (51,897--67,824) \\
	\hline
	\multirow{2}*{lsd-steepest-random} & 77,866 (75,315--81,017) & 51,453 (49,257--53,785) \\
	& 71,322 (68,623--76,002) & 48,234 (45,351--51,534) \\
	\hline
	\multirow{2}*{lscd-steepest-random} & 89,127 (82,350--101,152) & 62,892 (55,074--74,956) \\
	& 84,469 (74,393--96,069) & 60,203 (51,897--67,824) \\
	\hline
	\multirow{2}*{msls-steepest-random} & 75,048 (74,178--75,644) & 49,128 (48,311--49,620) \\
	& 68,211 (67,828--68,622) & 45,649 (45,075--46,032) \\
	\hline
	\multirow{2}*{ils-steepest-random} & 74,255 (73,754--74,535) & 48,312 (47,936--48,646) \\
	& 67,389 (67,008--67,744) & 44,616 (44,105--45,127) \\
	\hline
	\multirow{2}*{i$_a$ls-steepest-random} & 74,893 (74,137--75,593) & 48,969 (48,662--49,308) \\
	& 68,070 (67,676--68,496) & 45,505 (45,016--46,114) \\
	\hline
	\hline
	\multirow{2}*{lsns-nolsearch-from-greedy} & 73,777 (73,061--74,941) & 48,571 (47,475--49,299) \\
	& 68,001 (66,750--69,953) & 45,345 (44,229--46,356) \\
	\hline
	\multirow{2}*{lsns-nolsearch-from-lsearch} & 73,774 (73,221--74,547) & 48,391 (47,514--49,624) \\
	& 68,217 (66,814--70,149) & 45,249 (43,954--46,160) \\
	\hline
	\multirow{2}*{lsns-lsearch-from-greedy} & 73,609 (\textbf{72,859}--74,404) & 48,454 (47,562--49,040) \\
	& 66,987 (66,755--67,337) & 44,708 (43,869--46,217) \\
	\hline
	\multirow{2}*{lsns-lsearch-from-lsearch} & 73,607 (72,967--74,319) & 48,180 (\textbf{47,406}--49,137) \\
	& 67,112 (\textbf{66,585}--67,673) & 44,574 (\textbf{43,884}--45,715) \\
	\hline
	\multirow{2}*{evo-op1} & 75,752 (74,909--76,582) & 49,601 (48,387--50,805) \\
	& 68,656 (67,989--69,463) & 46,330 (45,247--47,714) \\
	\hline
	\multirow{2}*{evo-op2} & 75,766 (74,701--76,827) & 49,712 (48,367--50,408) \\
	& 68,713 (67,872--69,598) & 46,284 (45,364--47,530) \\
	\hline
	\caption{Average, minimum and maximum scores of found solutions}
\end{longtable}


\section{Visualizations}

\newcommand{\visualization}[3]{%
\begin{figure}[H]%
	\includegraphics[width=\linewidth]{results/best_#2_#1.pdf}%
	\vspace{-12mm}%
	\caption{Best #2 solution to #1 (#3)}%
\end{figure}%
}

\subsection{TSPA.csv}
\visualization{TSPA}{evo-op1}{74,909}
\visualization{TSPA}{evo-op2}{74,701}

\subsection{TSPB.csv}
\visualization{TSPB}{evo-op1}{67,989}
\visualization{TSPB}{evo-op2}{67,872}

\subsection{TSPC.csv}
\visualization{TSPC}{evo-op1}{48,387}
\visualization{TSPC}{evo-op2}{48,367}

\subsection{TSPD.csv}
\visualization{TSPD}{evo-op1}{45,247}
\visualization{TSPD}{evo-op2}{45,364}

\section{Source code}
The source code for all the experiments and this report is hosted on GitHub: \\
\url{https://github.com/RoyalDonkey/put-ec-tasks}

\section{Conclusions}

Evolutionary methods take considerably more effort to implement, and in their
most basic form perform quite average. However, their major strength is that,
once the framework has been established, it's relatively easy to swap and
compare different parameters and components.

I'm surprised that \textbf{op1} and \textbf{op2} performed pretty much the
same. I was expecting \textbf{op2} to be at least noticeably better. But it's
possible that this similarity is caused by \textbf{Common Edges} filling up the
vast majority of children, leaving little difference to be made by random or
repair operators. If I wasn't running late, I would test this hypothesis, but I
want to get this report out as soon as possible. Maybe I will find more insight
for report 10, because I will probably try to develop some kind of hybrid
evolutionary algorithm for it, basing it off of this report.

\end{document}
